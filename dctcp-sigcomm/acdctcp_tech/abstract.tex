\begin{abstract}

Multi-tenant datacenters are successful because tenants 
can seamlessly port their workloads, applications and services
to the cloud. Virtual Machine (VM) technology plays an integral role in this success
by enabling a diverse set of operating systems and
software to be run on a unified underlying framework. This
flexibility, however, comes at the cost of dealing with out-dated, inefficient, or 
misconfigured TCP stacks implemented in the VMs. This paper investigates if
administrators can take control of a VM's TCP congestion control
algorithm {\em without} making changes to the VM or network hardware.
We propose~\acdc{} TCP, a scheme that exerts fine-grained
control over arbitrary tenant TCP stacks by enforcing per-flow congestion
control in the vSwitch. Our scheme is light-weight, flexible, scalable and
can police non-conforming flows. Our evaluation shows the computational overhead
of~\acdc{} TCP is less than 4\% and implementing an administrator-defined congestion control algorithm
in the vSwitch (\ie{}, DCTCP) closely tracks its native performance, regardless of the VM's TCP stack. 


\end{abstract}
