\section{Introduction}
\label{intro}

Multi-tenant datacenters are a crucial component of today's computing ecosystem. Large providers, such as Amazon, Microsoft, IBM, Google and Rackspace, support
a diverse set of customers, applications and systems through their public cloud offerings. These offerings are successful in 
part because they provide efficient performance to a wide-class of applications running on a diverse set of platforms. Virtual
Machines (VMs) play a key role in supporting this diversity by allowing customers to run applications in a wide variety of 
operating systems and configurations.

And while the flexibility of VMs allows customers to easily move a vast array of applications into the cloud, that same flexibility inhibits the 
amount of control a cloud provider yields over VM behavior. For example, a cloud provider may be able to provide virtual networks or enforce rate limiting
on a tenant VM, but it cannot control the VM's TCP/IP stack. As the TCP/IP stack considerably impacts overall network performance, it 
is unfortunate that cloud providers cannot exert a fine-grained level of control over one of the most important components in the networking stack.

Without control over the VM TCP/IP stack, datacenter networks remain at the mercy of inefficient, out-dated or misconfigured TCP/IP stacks.
TCP behavior, specifically congestion control, has been widely studied and many issues have come to light when it is not optimized. For example,
network congestion caused by non-optimzed stacks can lead to loss, increased latency and reduced throughput. 
%As revenue increasingly is tied to 
%strict latency and bandwidth requirements from workloads such as big data analytics and search~\cite{alizadeh2011data,dean2013tail}, public cloud providers must ensure their
%network fabrics can provide tight service-level agreements and deadlines required by customer applications.

Thankfully, recent advances optimizing TCP stacks for datacenters have shown high throughput and low latency can be 
achieved through novel TCP congestion control algorithms. Works such as DCTCP~\cite{alizadeh2011data} and TIMELY~\cite{mittal2015timely} provide high
bandwidth and low latency by ensuring network queues in switches do not fill up. And while these stacks are deployed in many of today's 
private datacenters~\cite{singh2015jupiter,judd2015nsdi}, ensuring a vast majority of VMs within a public datacenter will update their TCP stacks
to a new technology is a daunting, if not impossible, task.

In this paper, we explore how operators can regain authority over TCP congestion control, regardless of the TCP stack
running in a VM. Our aim is to allow a cloud provider to utilize advanced TCP stacks, such as DCTCP, without having
control over the VM or requiring changes in network hardware. We propose implementing congestion control in the virtual switch
(vSwitch) running on each server. Implementing congestion control within a vSwitch has several advantages. 
First, vSwitches naturally fit into datacenter network virtualization architectures and are widely
deployed~\cite{Pfaff2015ovs}. Second, vSwitches can easily monitor and modify traffic passing through them. 
Today vSwitch technology is mature and robust, allowing for a fast, scalable,
and highly-available framework for regaining control over the network. 

%Since vSwitch technology supports software-defined networking (say OVS),
%implementing congestion control within the vSwitch can also naturally support advanced congestion control algorithms, such as centralized 
%or proactive schemes (cite). 

Implementing congestion control within the vSwitch has numerous challenges, however. First, in order to ensure adoption rates are high, the 
approach must work without making changes to VMs. 
Hypervisor-based approaches typically rely on rate limiters to limit VM traffic. Rate limiters implemented in
commodity hardware do not scale in the number of flows and software implementations incur high CPU overhead~\cite{radhakrishnan2014senic}. 
Therefore, limiting a VM's TCP flows in a fine-grained, dynamic nature
at scale (10,000's of flows per server~\cite{180302}) with limited computational overhead remains challenging. 
Finally, VM TCP stacks may differ in the features they support (\eg{}, ECN) or the congestion
control algorithm they implement, so a vSwitch congestion control implementation should work under a variety
of conditions. 

This paper presents Administrator Control over Datacenter TCP (\acdc{} TCP, or simply~\acdc{}), a new technology that implements 
TCP congestion control within a vSwitch to help ensure VM
TCP performance cannot impact the network in an adverse way. At a high-level, the vSwitch monitors all packets for a flow, modifies 
packets to support features not implemented in the VM's TCP stack (\eg{}, ECN) and reconstructs
important TCP parameters for congestion control.~\acdc runs the congestion control logic specified by an administrator and then enforces an intended
congestion window by modifying the receive window (\rwnd{}) on incoming ACKs. A policing
mechanism ensures stacks cannot benefit from ignoring~\rwnd{}.% and can also be used for non-TCP traffic.

Our scheme provides the following benefits. First,~\acdc allows
administrators to enforce a uniform, network-wide congestion control algorithm without changing VMs. When using congestion control algorithms tuned for 
datacenters, this allows for high throughput and low latency. Second,
our system mitigates the impact of varying TCP stacks running on the same fabric. This improves fairness and additionally
solves the ECN co-existence problem identified in production networks~\cite{wu2012tuning,judd2015nsdi}. 
Third, our scheme is easy to implement, computationally lightweight, scalable, and modular so that it is highly complimentary to
performance isolation schemes also designed for virtualized datacenter environments.
The contributions of this paper are as follows:
\begin{enumerate}
\item The design of a vSwitch-based congestion control mechanism that regains control over the VM's TCP/IP stack
without requiring any changes to the VM or network hardware. 
\item A prototype implementation to show our scheme is effective, scalable, simple to implement, and has~\crs{less than one percentage point} computational overhead in our tests.
%~\todo{We also provide a simple policing scheme to see if tenants are following rules. Add?} 
\item A set of results showing DCTCP configured as the host TCP stack provides nearly identical
performance to when the host TCP stack varies but DCTCP's congestion control is implemented in the vSwitch. We demonstrate how~\acdc{} can improve
throughput, fairness and latency on a shared datacenter fabric.
\end{enumerate}

The outline of this paper is as follows. Background and motivation are discussed in \cref{background}.~\acdc{}'s design is outlined in \cref{design} and
implementation in \cref{impl}. Results are presented in \cref{results}. Related work is surveyed in \cref{related}
before concluding.

