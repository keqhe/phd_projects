%%%%%%%%%%%%%%%%%%%%%%%%%%%%%%%%%%%%%%%%%%%%%%%%%%%%%%%%%%%%%%%%%%%%%%%%%%%%%%%%
\subsubsection{Protocols and Services}
\label{sec:protocols}

\begin{table}[!t]\small
\centering
\begin{tabular}{|c|r|r|}
	\hline
{\bf Cloud} &  {\bf Bytes} & {\bf Flows} \\
\hline
EC2 & 81.73 & 80.70 \\ \hline
Azure & 18.27 & 19.30  \\ \hline \hline
Total & 100 & 100 \\ \hline

\end{tabular}

% mysql> select  provider, sum(orig_ip_bytes + resp_ip_bytes) from uid_totalconnadd where direction = 'in' GROUP BY provider;
%+----------+------------------------------------+
%| provider | sum(orig_ip_bytes + resp_ip_bytes) |
%+----------+------------------------------------+
%| azure    |                       209622194641 |  18.27 %
%| ec2      |                       938013924091 | 81.73%
%+----------+------------------------------------+
%		Total					1147636118732
% flow count select provider, count(distinct uid) from uid_totalconnadd where direction = 'in' GROUP BY provider;
% azure 9869288  21.36%
% ec2 41287530  
% total 51156818


\caption{Percent of traffic volume and percent of flows associated with
    each cloud in the packet capture.}
\label{tab:cloudsplit}
\end{table}

\begin{table}[!t]\small
\centering
\begin{tabular}{|c||r|r||r|r||r|r|}
	\hline
& \multicolumn{2}{|c||}{\bf EC2} & \multicolumn{2}{|c||}{\bf Azure} &
\multicolumn{2}{|c|}{\bf Overall}\\ \hline
{\bf Protocol} &  Bytes & Flows &  Bytes &  Flows & Bytes &  Flows \\
\hline
ICMP & 0.01 & 0.03 & 0.01 & 0.18 & 0.01 & 0.06 \\ \hline
HTTP (TCP) & 16.26 & 70.45 & 59.97 & 65.41 & 24.24 & 69.48  \\ \hline
HTTPS (TCP) & 80.90 & 6.52 & 37.20 & 6.92 &72.94 & 6.60 \\ \hline
DNS (UDP) & 0.11 & 10.33 & 0.10 & 11.59 & 0.11 & 10.58 \\ \hline
Other (TCP) & 2.40 & 0.40 & 2.41 & 1.10 & 2.40 & 0.60 \\ \hline
Other (UDP) & 0.28 & 0.19 & 0.31 & 14.77 & 0.28 & 3.00   \\ \hline \hline
Total & 100 & 100 & 100 & 100  & 100 & 100 \\ \hline
%Total & 81.73 & 80.70 & 18.27 & 19.30  & 100 & 100 \\ \hline

\end{tabular}


\iffalse
\begin{tabular}{|c||r|r||r|r|}
\hline
\captureonedata & \multicolumn{2}{|c||}{\bf EC2} &
\multicolumn{2}{|c|}{\bf Azure} \\\hline
{\bf Protocol} & {\bf bytes (\%)} & {\bf flows (\%)} & {\bf bytes (\%)} &
{\bf flows (\%)} \\ \hline
ICMP&   0.01 & 0.05 & 0.01 & 0.22\\ \hline
HTTPS&  73.62 & 7.16 & 33.13 & 6.25\\ \hline
HTTP&   15.94 & 76.82 & 62.31 & 60.85\\ \hline
DNS&    0.12 & 13.20&   0.17 & 16.98\\ \hline
Other &    10.31 &  2.77& 4.38     & 15.70 \\ \hline
{\bf Total}&82 & &18 & \\ \hline
\hline
\capturetwodata & \multicolumn{2}{|c||}{\bf EC2} &
\multicolumn{2}{|c|}{\bf Azure} \\\hline
{\bf Protocol} & {\bf bytes (\%)} &  {\bf flows (\%)} & {\bf bytes (\%)} &
{\bf flows (\%)} \\ \hline
ICMP&   0.01 & 0.01 & $<$0.01 & $<$0.01\\ \hline
HTTPS&  16.31 & 9.61 & 40.77 & 21.57\\ \hline
HTTP&   82.70 & 49.42 & 51.82 & 47.00\\ \hline
DNS&    0.22 & 36.31&   0.39 & 12.51\\ \hline
Other &    0.76 &  4.65& 7.02     & 18.92 \\ \hline
{\bf Total}&97.7 & & 2.3& \\ \hline
\end{tabular}
\fi
% mysql> select  provider, sum(orig_ip_bytes + resp_ip_bytes) from uid_totalconnadd where direction = 'in' GROUP BY provider;
%+----------+------------------------------------+
%| provider | sum(orig_ip_bytes + resp_ip_bytes) |
%+----------+------------------------------------+
%| azure    |                       209622194641 |  18.27 %
%| ec2      |                       938013924091 | 81.73%
%+----------+------------------------------------+
%		Total					1147636118732
% flow count select provider, count(distinct uid) from uid_totalconnadd where direction = 'in' GROUP BY provider;
% azure 9869288  21.36%
% ec2 41287530  
% total 51156818
%icmp
%azure,icmp,30376335,17923
%ec2,icmp,111091019,14281
%total, icmp, 141467354, 32204

%other udp
%mysql> select provider, proto, sum(orig_ip_bytes + resp_ip_bytes) AS bytes, count(distinct uid) from uid_totalconnadd
%    -> WHERE direction = 'in'
%    -> AND proto = 'udp'
%    -> AND resp_p != '53'
%    -> GROUP BY provider WITH ROLLUP;
%+----------+-------+------------+---------------------+
%| provider | proto | bytes      | count(distinct uid) |
%+----------+-------+------------+---------------------+
%| azure    | udp   |  649279919 |             1458078 |
%| ec2      | udp   | 2613832393 |               77649 |
%| NULL     | udp   | 3263112312 |             1535727 |
%+----------+-------+------------+---------------------+
% other tcp
% mysql> select provider, proto, sum(orig_ip_bytes + resp_ip_bytes) AS bytes, count(distinct uid) from uid_totalconnadd
%    -> WHERE direction = 'in'
%    -> AND proto = 'tcp'
%    -> AND resp_p != '443'
%    -> AND resp_p != '80'
%    -> GROUP BY provider WITH ROLLUP;
%+----------+-------+-------------+---------------------+
%| provider | proto | bytes       | count(distinct uid) |
%+----------+-------+-------------+---------------------+
%| azure    | tcp   |  5041918348 |              109052 |
%| ec2      | tcp   | 22735102034 |              205169 |
%| NULL     | tcp   | 27777020382 |              314221 |
%+----------+-------+-------------+---------------------+

%total dns
%mysql> select provider, resp_p, sum(orig_ip_bytes + resp_ip_bytes) AS bytes, count(distinct uid) from uid_totalconnadd
%    -> WHERE direction = 'in'
%    -> AND proto = 'udp'
%    -> AND resp_p = '53'
%    -> GROUP BY provider WITH ROLLUP;
%+----------+--------+------------+---------------------+
%| provider | resp_p | bytes      | count(distinct uid) |
%+----------+--------+------------+---------------------+
%| azure    |     53 |  205574625 |             1144259 |
%| ec2      |     53 | 1020349800 |             4266683 |
%| NULL     |     53 | 1225924425 |             5410942 |
%+----------+--------+------------+---------------------+

%total http
%mysql> select provider, resp_p, sum(orig_ip_bytes + resp_ip_bytes) AS bytes, count(distinct uid) from uid_totalconnadd WHERE direction = 'in'  AND proto = 'tcp' AND resp_p = '80'  GROUP BY provider WITH ROLLUP;
%+----------+--------+--------------+---------------------+
%| provider | resp_p | bytes        | count(distinct uid) |
%+----------+--------+--------------+---------------------+
%| azure    |     80 | 125715559055 |             6456413 |
%| ec2      |     80 | 152476406498 |            29086853 |
%| NULL     |     80 | 278191965553 |            35543266 |
%+----------+--------+--------------+---------------------+
%total https
%select provider, resp_p, sum(orig_ip_bytes + resp_ip_bytes) AS bytes, count(distinct uid) from uid_totalconnaddWHERE direction = 'in' AND proto = 'tcp' AND resp_p = '443'GROUP BY provider WITH ROLLUP;
%+----------+--------+--------------+---------------------+
%| provider | resp_p | bytes        | count(distinct uid) |
%+----------+--------+--------------+---------------------+
%| azure    |    443 |  77979486359 |              683563 | 
%| ec2      |    443 | 759057142347 |             2690635 | 
%| NULL     |    443 | 837036628706 |             3374198 |
%+----------+--------+--------------+---------------------+


\caption{Percent of traffic volume and percent of flows associated with each  
    protocol in the packet capture.}
\label{tab:proserv}
\end{table}

We first examine the fraction of bytes and flows in the \capturedata that are
associated with each cloud (\tabref{tab:cloudsplit}). We only consider flows
that were initiated within the university and destined for EC2 or Azure.
We observe that the majority of cloud traffic, both as measured by volume and
number of flows, is EC2-related: 81.73\% of bytes (80.70\% of flows)
are associated with EC2, while Azure accounts for 18.27\% of bytes
(19.30\% of flows).

Next, we use the \capturedata to study the application-layer protocols used by cloud tenants.
\tabref{tab:proserv} shows the
percentage of bytes (and flows) using a specific protocol relative to the
total number of bytes (and flows) for EC2, Azure, and the capture as a whole.

We observe that more than $99\%$ of bytes in the \captureonedata 
are sent and received using TCP, with less than $1\%$ of bytes
associated with UDP or ICMP.  The vast majority of this TCP traffic is
HTTP and HTTPS. The proportion of HTTPS traffic is far higher than
that seen for general web services in the past (roughly
6\%~\cite{agarwal2010endre}); as we will show later, HTTPS traffic is dominated
by cloud storage services. Interestingly, the majority of Azure's TCP traffic
is HTTP (59.97\%) while the majority of EC2's TCP traffic is HTTPS (80.90\%)  

The breakdown by flow count is less skewed
towards TCP, with UDP flows accounting for 14\% of flows in the
\captureonedata. This is largely due to DNS queries, which account for
11\% of flows but carry few bytes.

As one would expect, public IaaS clouds are also used for non-web-based
services.  In the \captureonedata, we find a small fraction of non-HTTP(S)
TCP traffic and non-DNS UDP traffic going to both EC2 and Azure. 
This traffic includes SMTP, FTP, IPv6-in-IPv4, SSH, IRC, and other traffic that Bro
could not classify.


\tightparagraph{Summary and implications} While we analyze a single
vantage point, our measurements suggest that web
services using HTTP(S) represent an important set of WAN-intensive
cloud tenants. The extent to which compute-intensive workloads (that
may not result in a large impact on network traffic) are prevalent as
cloud tenants remains an interesting open question.  In the following
sections we dig into what tenants are hosting web services on public
clouds as well as diving deeper into their traffic patterns.


