\section{Related Work}
\label{sec:related}
% Recently there has been several publications that attempt to characterize the
% performance of commercially available Openflow switches and proposed solutions
% to circumvent these performance issues. In the context of the work presented
% here we survey prior work on performance studies of commercial switches and
% from a solution perspective we provide the context for two solutions: Rule
% offloading and Multipath routing. 

A few prior studies have considered SDN switch performance. However, 
they either focus on narrow issues, do not offer
sufficient in-depth explanations for observed performance issues, or do
not explore implications on applications that require tight
control of latency.  Devoflow~\cite{devoflow} showed that the rate of statistics
gathering is limited by the size of the flow table and that statistics
gathering negatively impacts flow setup rate.  More recently, two
studies~\cite{oflops,ucsdpaper} provided a more in-depth look into
switch performance across various vendors.  In~\cite{oflops}, the
authors evaluate 3 commercial switches and observed that switching
performance is vendor specific and depends on applied operations,
forwarding table management, and firmware. In ~\cite{ucsdpaper}, the
authors also studied 3 commercial switches (HP Procurve, Fulcrum,
Quanta) and found that delay distributions were distinct, mainly due
to variable control delays.  Our work is complementary with, and more
general, than these results. We provide in-depth characterization of the impact of
rule priority structures. We also provide
low-level explanations of the latency causes.

% However all of the above work was limited in
% scope and only addressed control path delays \cite{ucsdpaper}, impact
% of statistics gathering \cite{devoflow} and the performance of general
% openflow command operations \cite{oflops}. In our work we design a
% measurement set up that is similar to the one used in \cite{ucsd} and
% attempt to comprehensively characterize the different components of
% the flow set up latency on commercially available Broadcom and Intel
% switch platforms.

% One of the first papers to evaluate the performance of a commercial switch was
% Devoflow\cite{devoflow}.  
% The evaluation was done on the HP Procurve 5406z1 switch and the study was
% aimed primarily at analyzing the various factors that  
% impact statistics gathering within the switch. The authors show that currently
% Openflow switches are the primary bottleneck to support the necessary  
% flow set up and statistics gathering rates for a typical data center network. 
% It was observed that 

% OFLOPS \cite{oflops} describes a software framework that can be used to
% evaluate the performance of Openflow switches. More recently \cite{ucsdpaper}
% designed a software openflow switch emulator. Both the OFLOPS framework and
% the switch emulator work conducted benchmarking  
% studies on openflow switch performance. The OFLOPS framework was used to
% evaluate 3 commercial switches and the authors observed that switching
% performance is vendor specific and depends on applied operations, forwarding
% table management and firmware. The \cite{ucsd} work also showed that for 3
% commercial switches (HP Procurve, Fulcrum, Quanta) there was distinct delay
% distributions that are due to control path delays.  However all of the above
% work was limited in scope and only addressed control path delays
% \cite{ucsdpaper}, impact of statistics gathering \cite{devoflow} and the
% performance of general openflow command operations \cite{oflops}. In our work
% we design a measurement set up that is similar to the one used in \cite{ucsd}
% and attempt to comprehensively characterize the different components of the
% flow set up latency on commercially available Broadcom and Intel switch
% platforms.  

% To address the performance issues of the switches several solutions have been
% proposed. Some approaches are implemented in the control plane or hypervisor  
% \cite{Cassado}\cite{Seawall} while others primarily involve the data plane.

% Our measurement study clearly shows that the solution choice in the data plane
% is not trivial since  
% the switch is the performance bottleneck. Here we focus on data plane related
% solutions to reduce flow set up latencies.  
Some studies have consider approaches to mitigate the overhead of SDN
rule matching and processing. DevoFlow~\cite{devoflow} presents
a rule cloning solution which reduces the number of controller
requests being made by the switch by having the controller set up
rules on aggregate or elephant flows. Mazu's techniques are largely
complementary to this. 
% A clone flag is used to indicate if the flow rule should remain
               % as an aggregate in the flow table or 
% if it needs to be split into microflows. The ability to split into microflows
% allows greater visibility into  
% these flows from a management perspective while at the same time reducing the
% number of rule set up requests to the controller. The action part of the flow
% table in the switch can also implement explicit multipath routing and rapid
% rerouting by manipulating the  
% rule priorities within the switch. These proposed solutions were evaluated
% using simulation studies and show some promise but requires modifications to
% current switch operations. Our measurement study suggests that the
% implementation of these approaches on  
% current switches especially the use of priorities may further impact switch performance. 
DIFANE \cite{difane} reduces flow set up latency by splitting
pre-installed wild card rules among multiple switches and therefore
all decisions are still made in the data plane.  However this approach does not
apply for the kind of applications we are targeting that need to make fast,
frequent updates/modifications to data plane state. 
%\aditya{keqiang: check this}
% reduces the global visibility of flow states and the statistics needed
% for flow management.  \marina{add a sentence about why our rule
%   offload and multipath can still maintain global vissibility of the
%   network}. 
vCrib~\cite{minlanvcrib} automatically partitions and places rules at both hypervisors and
switches, but their goal is to reduce computational load on the
host hypervisor, while we wish to reduce path setup latency by enabling fast
parallel execution of updates.
% \marina{add a sentence about how our rule space partition and placement is
% different from Vcrib,  
% in terms of reducing the number of rules per partition and number of
% partitions}.  
Lastly, Dionysus~\cite{dionysus} optimally schedules a set of rule updates 
while maintaining desirable consistency properties (e.g., no loops and no
blackholes). Dionysus assumes the target network state is given, while Mazu
computes a target network state that spreads rule updates among more switches
in order to increase parallelism. Thus, we believe that Dionysus and Mazu are
complementary.

% LocalWords:  SDN Devoflow Procurve Mazu's DIFANE pre hypervisors hypervisor
