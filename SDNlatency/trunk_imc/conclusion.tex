\section{Conclusion}
\label{sec:conclusion}
Critical SDN applications such as fast failover and mobility demand
tight interaction between switch control and data planes. However, our
measurements across three OpenFlow-based switches show that the
latencies underlying the generation of control messages (\packetin's)
and execution of control operations (\flowmod's) can be quite high, and
variable. We find that the underlying causes are linked both to
software inefficiencies, as well as pathological interactions between
switch hardware properties (shared resources and how forwarding rules
are organized) and the control operation workload (the order of
operations issues, and concurrent switch activities).  To mitigate the
challenges these latencies create for SDN in supporting critical
management applications, we present a system called Mazu. Mazu
bypasses switch processing of \packetin\ messages by redirecting
unmatched packets to a proxy. To reduce the latency of
\flowmod\ operations, Mazu employs: (1) \emph{flow engineering}, a
mechanism to allow management applications to take path setup latency
as a second objective, and, (2) {\em rule offloading}, which computes
strategies for opportunistically offloading portions of forwarding
state to be installed at a switch to other switches downstream from
it. Our evaluation shows that these mechanisms can tame flow setup
latencies, thereby enabling SDN-based control of critical
applications.

% are very effective. As part of our
% future work, we would like to incorporate the Mazu framework in controller
% platforms such as Floodlight.

% LocalWords:  SDN failover Mazu
