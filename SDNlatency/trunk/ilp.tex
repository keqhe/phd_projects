

\section{Flow Engineering Formulation}
\label{app:floweng}
In this appendix we present an integer linear program for flow engineering
(\secref{s:floweng}). We assume the SDN application performs traffic 
engineering in a data center with the goal of minimizing the maximum link
utilization. \tabref{notations} lists the notations used in the formulation.

%\subsection{Notation}

\begin{table}[h!]
\centering
\small
\begin{tabular}{c|p{0.7\columnwidth}}
\hline
$S$ & Set of all switches \\
$S_{ToR}$ & Set of all top-of-rack (ToR) switches \\
$\tau_{u}, \forall u \in S$ & Maximum number of flow entries in the switch $u$ \\
$E$ & Set of all physical links (between adjacent switches) \\
$C_e, \forall e \in E$ & Capacity of individual links \\
$F_{uv}$ & Set of all flows from $u$ to $v$ where $u,v \in S_{ToR}$ \\
$P_{uv}$ & Set of paths from switch $u$ to switch $v$ \\
$K _{uv}$ & Number of paths from switch $u$ to $v$ where $u,v \in S_{ToR}$ \\
$P^k _{uv}$ & Set of links of $k^{th}$ path from device $u$ to $v$ where $u,v \in S_{ToR}$ \\
$T ^f _{uv}$ & Traffic volume from $u$ to $v$ of flow $f$ where $u,v \in S_{ToR}$ \\
$util$ & Maximum link utilization \\
$D_{u} \forall u \in S$ & Number of low priority rules in switch $u$ \\
$I ^{fk} _{uv}$ & Indicator variable denoting that flow $f$ from $u$ to $v$  takes the $k^{th}$ path \\
\hline
%\tabincell{c}{HP Procurve \\8212zl} & 666Mhz & 256MB & $\approx$1000 & Modularity \\ \hline
%\hline
\end{tabular}
\topcompactcaption{Notations used in flow engineering formulation}{\label{notations}}
\end{table}

\iffalse
\begin{itemize}
\item $S$ - Set of all switches
\item $S_{ToR}$ - Set of all ToR switches
\item $\tau_{u}, \forall u \in S$ - Maximum number of flow entries in the switch $u$
\item $E$ - Set of all physical links (between two adjacent devices)
\item $C_e, \forall e \in E$ - capacity of individual links
\item $F_{uv}$ - set of all flows from $u$ to $v$ where $u,v \in S_{ToR}$
\item $P_{uv}$ - Set of paths from device $u$ to device $v$
\item $K _{uv}$ - Number of paths from device $u$ to $v$ where $u,v \in S_{ToR}$ 
\item $P^k _{uv}$ - Set of links of $k^{th}$ path from device $u$ to $v$ where $u,v \in S_{ToR}$
\item $T ^f _{uv}$ - Traffic volume from $u$ to $v$ of flow $f$ where $u,v \in S_{ToR}$
\item $util$ - maximum link utilization
\item $I ^{fk} _{uv}$ - Indicator variable denoting that flow $f$ from $u$ to $v$  takes the $k^{th}$ path.
%\item $entries$ - maximum number of flow rules in any switch
\end{itemize}
\fi

\subsection{Step 1 - Minimize ``utilization''}
In the first step, we minimize the maximum link utilization. 
We represent the network as a graph \textit{G = (V,E)}, where each vertex is
a switch and each edge is a physical link. Let $T ^f _{uv}$ be the
traffic volume between $(u,v)$ of flow \textit{f}, $P^k _{uv}$ be the set of
links on the $k^{th}$ path between (\textit{u,v}). 

We have two constraints: First, the total traffic volume from an edge should
not exceed its capacity, $C_e$, times the link utilization limit, $util$. We
use a binary indicator variable  $I ^{fk} _{uv}$ to show whether or not the 
flow $f$ is on path $k$.

	
\begin{equation}
    \sum _{u,v \in S_{ToR},~f \in F_{uv},~k  \in P_{uv}~s.t.~e \in P^k _{uv}} T ^f _{uv} *  I^{fk}_{uv} \leq util \times C_e 
\end{equation}
	
	
\noindent Second, each flow $f$ should be routed from a single path.

\begin{equation}
    \sum _{k  \in P_{uv}} I^{fk}_{uv} = 1 ,\quad \quad \forall u,v \in S_{ToR},~ f \in F_{uv}
\end{equation}
  

The objective is to minimize \textit{util}, the maximum link utilization, under the above two constraints. 

\subsection{Step 2 - Minimize ``rule setup latency''}
Our measurements show that flow installation latency has a linear relation
with the burst size of the rules (i.e., the number of rules to be installed), 
and, if rules are of a higher (or lower, depending on switch vendor) priority,
installation latency increases with the number of rules that need to be 
displaced. Therefore, the second step is minimize the maximum number of
rules in any switch. %, as number of rules has a linear relationship with installation
%latency. Following are three constraints.
Flow installation latency can be defined as 
\begin{equation*} 
%latency_{u} = \max(a, (b +c * Disp_u(Pri(f))))
latency_{u} = (a_{u}  + c_{u}* D_{u}) *\tau_{u}
\end{equation*}
where $a_u$ and $c_u$ are constants for switch $u$ derived from
measurements.
%, and $Disp_u(Pri(f))$ is the number of rules at $u$ that will be
%displaced by the rule for flow $f$.
%     .\fixme{seem to be inconsistent with FE part} Flow installation latency
%     can be define as \begin{equation*} latency_{u} = a_{u} * \tau_{u} + c_{u}
%     \end{equation*} , where $a_{u}$ and $c_{u}$ are constants for switch $u$.


There are three constraints:
First, traffic traversing an edge $e$ does not exceed the $C_e$
multiplied by 1 $+$ $\delta$ times the $util^*$. $util^*$ is the maximum link
utilization we got from the step 1, and  $\delta$ defines an upper limit on
maximum link utilization.

\begin{equation}
    \sum _{u,v \in S_{ToR},~f \in F_{uv},~k  \in P_{uv}~s.t.~e \in P^k _{uv}} T ^f _{uv} *  I^{fk}_{uv} \leq (1 + \delta) \times util^* \times C_e 
\end{equation}

\noindent Second, constraint is same as step 1.
  
\begin{equation}
      \sum _{k  \in P_{uv}} I^{fk}_{uv} = 1 ,\quad \quad \forall u,v \in S_{ToR},~ f \in F_{uv}
\end{equation}

\noindent Third, the number of flows passing through a switch $s$ is less than the number of maximum rules in any switch in the network.
\begin{equation}
                       \sum_{u,v \in S_{ToR}, ~f \in F_{uv}, ~k \in P_{uv} ~s.t.~ s\in P^k_{uv} } I^{fk}_{uv} \leq \tau_{u} \quad \quad \forall s \in S
\end{equation}
The objective is to minimize the $latency_{u}$, the maximum flow installation latency, under above three constraints.   
  

