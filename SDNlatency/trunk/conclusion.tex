\section{Conclusion}
\label{sec:conclusion}
Critical SDN applications such as fast failover and fine-grained traffic engineering demand
tight interaction between switch control and data planes. 
However, our measurements across four SDN switches show that the
latencies underlying the generation of control messages and execution of control operations can be quite high, and
variable. We find that the underlying causes are linked to
software inefficiencies, as well as pathological interactions between
switch hardware properties (shared resources and how forwarding rules
are organized) and the control operation workload (the order of
operations issued, and concurrent switch activities). 
%Our measurements highlight the need for careful design of next generation switch silicon 
%and software in order to fully utilize the power of SDN.
Given the software ``glue'' between switch control and data planes and 
the long upgrade cycles of switch hardware, we believe the latencies we identify will continue to 
manifest even in next generation switches. 
Our measurement study highlights the need for careful design of 
future switch silicon and software in order to fully utilize the power of SDN.

\iffalse
Finally, to mitigate the
challenges these latencies create for SDN in supporting critical
management applications, we present three measurement-driven techniques.
% (1) \emph{flow engineering}, a
%mechanism to allow management applications to take path setup latency
%as a second objective, and, (2) {\em rule offloading}, which computes
%strategies for opportunistically offloading portions of forwarding
%state to be installed at a switch to other switches downstream from
%it. 
Our evaluation shows that these mechanisms can tame flow setup
latencies effectively, thereby enabling SDN-based control of critical
applications.
\fi
% are very effective. As part of our
% future work, we would like to incorporate the Mazu framework in controller
% platforms such as Floodlight.

% LocalWords:  SDN failover Mazu
